%!TEX program = xelatex

\documentclass{crr}

\entite{Telecom Nancy}
\equipe{LAGARDE - JACOTOT}
\projet{RS1}

\date{23 octobre 2019} % Respecter le format suivant : «1 janvier 2019»
\lieu{TELECOM Nancy, S0.7}
\heure{16h00} % Respecter le format suivant : «13h45»
\duree{1} % La durée est en heure
\motif{Réunion technique}

\begin{reunion}
        \begin{contexte}
                \begin{participants}
                        \begin{presents}
                                \present{Henry Lagarde}
                                \present[redacteur]{Louis Jacotot}
                        \end{presents}
                        %\begin{excuses}
                                %\excuse{}
                        %\end{excuses}
                        %\begin{nonexcuses}
                                %\nonexcuse{}
                        %\end{nonexcuses}
                \end{participants}
                \approbation
        \end{contexte}
        \begin{ordredujour}
                \ajouter{Discussion technique}
        \end{ordredujour}
        \begin{echanges}
                \ajouter{Henry Lagarde présente le travaille qu'il a effectué}
                \ajouter{Les membres se répartissent le travail à effectuer
                pendant la semaine de vacances}
        \end{echanges}
        \begin{decisions}
                \ajouter{Louis Jacotot s'occupe de la création des processus
                fils et de l'exécution de \emph{gcc}}
                \ajouter{Henry Lagarde planifie l'intégration des premières
                options proposées}
        \end{decisions}
\end{reunion}

