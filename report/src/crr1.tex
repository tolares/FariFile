%!TEX program = xelatex

\documentclass{crr}

\entite{Telecom Nancy}
\equipe{LAGARDE - JACOTOT}
\projet{RS1}

\date{9 octobre 2019} % Respecter le format suivant : «1 janvier 2019»
\lieu{TELECOM Nancy, salle de travail}
\heure{16h00} % Respecter le format suivant : «13h45»
\duree{1} % La durée est en heure
\motif{Réunion de définition}

\begin{reunion}
        \begin{contexte}
                \begin{participants}
                        \begin{presents}
                                \present{Henry Lagarde}
                                \present[redacteur]{Louis Jacotot}
                        \end{presents}
                        %\begin{excuses}
                                %\excuse{}
                        %\end{excuses}
                        %\begin{nonexcuses}
                                %\nonexcuse{}
                        %\end{nonexcuses}
                \end{participants}
                \approbation
        \end{contexte}
        \begin{ordredujour}
                \ajouter{Définition des rôles dans l'équipe}
                \ajouter{Création et organisation du dépôt git \faGitlab}
                \ajouter{Discussion autour du sujet}
        \end{ordredujour}
        \begin{echanges}
                \ajouter{Henry Lagarde propose de prendre la responsabilité du
                        projet}
                \ajouter{Louis Jacotot propose d'utiliser des indentations de
                        quatre espaces pour éviter les conflits entre éditeurs}
        \end{echanges}
        \begin{decisions}
                \ajouter{\important{Henry Lagarde} est \important{responsable du
                        projet}}
                \ajouter{Louis Jacotot rédige les compte-rendus de réunions}
                \ajouter{Les indentations se composent de quatre espaces}
        \end{decisions}
\end{reunion}

