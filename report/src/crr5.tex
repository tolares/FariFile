%!TEX program = xelatex

\documentclass{crr}

\entite{Telecom Nancy}
\equipe{LAGARDE - JACOTOT}
\projet{RS1}

\date{16 décembre 2019} % Respecter le format suivant : «1 janvier 2019»
\lieu{Réunion à distance}
\heure{19h00} % Respecter le format suivant : «13h45»
\duree{1/2} % La durée est en heure
\motif{Réunion technique}

\begin{reunion}
        \begin{contexte}
                \begin{participants}
                        \begin{presents}
                                \present{Henry Lagarde}
                                \present[redacteur]{Louis Jacotot}
                        \end{presents}
                        %\begin{excuses}
                                %\excuse{}
                        %\end{excuses}
                        %\begin{nonexcuses}
                                %\nonexcuse{}
                        %\end{nonexcuses}
                \end{participants}
                \approbation
        \end{contexte}
        \begin{ordredujour}
                \ajouter{Discussion avec M. Eisenbarth}
                \ajouter{Correction des fonctionnalités}
        \end{ordredujour}
        \begin{echanges}
                \ajouter{Les membres corrigent la fonctionnalité Java}
                \ajouter{Les membres trouvent l'origine du problème sur la
                fonctionnalité Json}
        \end{echanges}
        \begin{decisions}
                \ajouter{Henry Lagarde corrige la fonctionnalité Json}
                \ajouter{Les membres rédigent un compte-rendu de projet pour
                limiter la perte de points sur la notation si le programme
                dysfonctionne après les derniers correctifs}
        \end{decisions}
\end{reunion}

